%% Preámbulo
%% ----------------------------------------------------------------------------
\documentclass[12pt,a4paper]{practice}
\usepackage[spanish]{babel}
\usepackage[utf8]{inputenc}
\usepackage[T1]{fontenc}
\usepackage{enumitem}
\usepackage{hyperref}
\usepackage{luximono}
\usepackage{textcomp}
\usepackage{graphicx}
\usepackage{multirow}
\usepackage{tabularx}
\usepackage{ragged2e}
\usepackage{amstext}
\usepackage{caption}
\usepackage{charter}

%% Settings
%% ----------------------------------------------------------------------------
% hyperref
\hypersetup{
    pdftitle={Atm\'{o}sferas Estelares - Pr\'{a}ctica 1},
    pdfauthor={Mart\'{i}n Josemar\'{i}a Vuelta Rojas},
    pdfpagelayout=OneColumn,
    pdfnewwindow=true,
    pdfdisplaydoctitle=true,
    pdfstartview=XYZ,
    plainpages=false,
    unicode=true,
    bookmarksnumbered=true,
    bookmarksopen=true,
    bookmarksopenlevel=3,
    breaklinks=true,
    colorlinks=true,
    pdfborder={0 0 0}
}

% graphicx
\graphicspath{{resources/img/}}

% caption
\captionsetup{
    labelfont=bf,
    textfont=it,
    justification=centering,
    width=0.9\textwidth,
}

% tabularx rule width
\setlength{\arrayrulewidth}{1.5pt}

%% Definicion de comandos
%% ----------------------------------------------------------------------------
\makeatletter

%% Fuente de ancho fijo
\renewcommand{\ttdefault}{lmtt}
\renewcommand{\spanishtablename}{Tabla}

\makeatother

%% Documento
%% ----------------------------------------------------------------------------
\begin{document}
    %% Titulo, autor y resumen --------------------------------------------------
    \logo{unmsm.png}
    \university{
        Universidad Nacional Mayor de San Marcos\\
        {
            \scriptsize{
                \textit{
                    \textup{
                        Universidad del Perú, Decana de América
                    }
                }
            }
        }
    }
    \course{Atmósferas Estelares}
    \title{Práctica N\textdegree\ 1}
    \maketitle

    \begin{problem}\label{prob:1}
        Determine la \emph{longitud de onda equivalente} para cada una de las curvas de sensibilidad dadas en la tabla \ref{table:p1_curvas_sensibilidad}.
    \end{problem}

    \begin{problem}\label{prob:2}
        Calcular la \emph{longitud de onda efectiva} del filtro $V$ para el flujo de un cuerpo negro con las siguientes temperaturas: $T = 25000 K$, $T = 10000 K$, $T = 5000 K$.
    \end{problem}

    \begin{problem}\label{prob:3}
        Se tiene el flujo de cuerpo negro observado con un receptor cuya curva de sensibilidad es la curva de sensibilidad del filtro $V$ del sistema fotometrico $UBV$. Determine la longitud de onda del flujo monocromático efectivo para las temperaturas $T = 25000 K$, $10000 K$ y $5000 K$.

        \begin{recommendation}
            Considerando que el flujo monocromático efectivo está dado por

                $$\left\langle B\right\rangle = \frac{\displaystyle{\int_{0}^{\infty} V_{\lambda} B_{\lambda} \left(T\right) \mathrm{d}\lambda}}{\displaystyle{\int_{0}^{\infty} V_{\lambda} \mathrm{d}\lambda}},$$

            determine para que valores de $\lambda$ se da la igualdad $\left\langle B\right\rangle = B_{\lambda} \left(T\right)$.
        \end{recommendation}
    \end{problem}

    \begin{problem}\label{prob:4}
        ¿Cuál es el cambio $\delta V$ en la magnitud $V$ del sistema fotmétrico $UBV$ que produce un cambio $\delta\lambda$ en la \emph{longitud de onda efectiva} calculada en \ref{prob:2}?

        Calcular $\delta\lambda = {\lambda}_{eq} - {\lambda}_{eff}$.

        \begin{recommendation}
            Si $V = -2.5 \log f_V + C$ tomar $f_V \simeq B\left(T\right) $, $T=T\left({\lambda}_{eff}\right)$; suponer $B_{\lambda}\ \alpha\ {\lambda}^{-v} \mathrm{e}^{-\frac{hc}{\lambda k T}}$ (Ley de Wien); calcular $\left(\frac{\mathrm{d} \ln {f}_{\lambda}}{\mathrm{d}\lambda}\right)_{\lambda = {\lambda}_{eq}}$
        \end{recommendation}
    \end{problem}

    \begin{problem}\label{prob:5}
        ¿Cuál es el cambio pocentual en $f_V$ que representa el cambio $\delta V$ calculado en \ref{prob:4}?.
    \end{problem}

    \begin{problem}\label{prob:6}
        Usando las tablas adjuntas (Referencia del COX) y considerando los tipos espectrales \emph{O9}, \emph{B0}, \emph{B2}, \emph{B5}, \emph{A0}, \emph{A5}, \emph{F0}, \emph{F5}, \emph{G0}, \emph{G5}, \emph{K0}, \emph{K5} y \emph{M0}.

        \begin{ppart}\label{prob:6:a}
            Graficar $M_V$ vs. $\left(B-V\right)_{0}$ para cada clase de lumninosidad V, III y I.
        \end{ppart}

        \begin{ppart}\label{prob:6:b}
            Calcular la temperatura de color $T_{BV}$ para los tipos espectrales de la secuencia principal.
        \end{ppart}

        \begin{ppart}\label{prob:6:c}
            Usando los datos de la tabla \ref{table:p6_color_int_UB} de \emph{colores intrínsecos} $\left(U-B\right)_O$, calcular las temperaturas de color $T_{UB}$.
        \end{ppart}

        \begin{ppart}\label{prob:6:d}
            Comparar los tipos de colores $T_{UV}$ y $T_{VB}$ de los tipos espectrales dados en \ref{prob:6:c}.
        \end{ppart}

        \begin{ppart}\label{prob:6:e}
            Comparar las temperaturas de color $T_{BV}$ y $T_{UV}$ de los espectrales dados en \ref{prob:6:b} con las temperaturas espectrales de la tabla adjuntada.

            \begin{itemize}
                \item ¿Cuál de las tempraturas de color parece aproximarse mejor desde un punto de vista cuantitativo a la temperatura afectiva?
                \item ¿Cúal de las tempraturas de color parece aproximarse mejor desde un punto de vista cualitativo a la temperatura efectiva?
            \end{itemize}
        \end{ppart}
    \end{problem}

    \begin{problem}\label{prob:7}
        Dibuja un diagrama $\left(V-B\right)_O$ vs. $\left(B-V\right)_O$ pero los tipos espectrales del ejercicio \ref{prob:6:b}. Superponer en el mismo diagrama la relación color-color del cuerpo negro. El cuerpo negro ajusto bien las observaciones.
    \end{problem}

    \begin{problem}\label{prob:8}
        Para los tipos espectrales de la tabla \ref{table:p6_color_int_UB}

            \begin{ppart}\label{prob:8:a}
                Calcular el coeficiente $\Phi$.
            \end{ppart}

            \begin{ppart}\label{prob:8:b}
                Calcular $\Phi$ suponiendo $T_{BV} = T_{UB} = T_{eff}$.
            \end{ppart}

            \begin{ppart}\label{prob:8:c}
                Calcular $\Phi$ usando sólo $T_{BV}$.
            \end{ppart}

            \begin{ppart}\label{prob:8:d}
                Calcular $\Phi$ usando sólo $T_{UB}$.
            \end{ppart}
    \end{problem}

    \newpage
    \begin{problem}\label{prob:9}
        \begin{ppart}\label{prob:9:a}
            Calcular un diagrama $\log (L/L_{\odot})$ vs $\log (T_{FF})$ para $\log \left(R/R_{\odot}\right) = -3, -2, -1,$ $0, +1, +2$.
        \end{ppart}

        \begin{ppart}\label{prob:9:b}
            Mostrar que la correccion bolometrica entre magnitudes relativas es igual a la correccion bolometrica entre magnitudes absolutas.
        \end{ppart}
    \end{problem}

    \begin{problem}\label{prob:10}
        Discutir porque la correccion bolometrica es funcion proncipalmente de la temperatura efectiva y no del radio.

        En esas condiciones la correccion bolometrica depende de la clase de luminosidad?¿Porque?
    \end{problem}

    \begin{problem}\label{prob:11}
        Si $\log f_{\lambda} = a\log\lambda + b$ en un intervalo $(\lambda_{1}, \lambda_{2})$ donde $f_{\lambda}$ es el flujo monocromatico observado de una estrella, mientras que el gradiente de color en ese intervalo es $\Phi _{\lambda_{1},\lambda_{2}} = \left(5+a\right)\lambda$. Donde $\lambda$
    \end{problem}

    \begin{problem}\label{prob:12}
        \begin{ppart}\label{prob:12:a}
            Mostrar que el gradiente de color de una radiacion de cuerpo negro esta dado por
            $$
                \Phi = \frac{c_2}{T}\left(1 - \mathrm{e}^{-\frac{c_2}{\lambda T}}\right)
            $$

            $$
                c_2 = \frac{hc}{k} \approx 1.43883\ \mathrm{cm \cdot K}
            $$
        \end{ppart}

        \begin{ppart}\label{prob:12:b}
            ¿En que region $\Phi = \frac{c_2}{\lambda T}$?
        \end{ppart}

        \begin{ppart}\label{prob:12:c}
            Calcular $\Phi$ usando $\lambda = \frac{1}{2}\left(\lambda_B + \lambda_V\right)_{eq}$ y $T = T_{BV}$ para los tipos espectrales de \ref{prob:6:d}.
        \end{ppart}
    \end{problem}

    \begin{problem}\label{prob:13}
        Transofrmar
            \begin{itemize}
                \item $M_{V}$ en $M_{bol}$.
                \item $M_{V}$ en $T_{BV}$.
            \end{itemize}

            Para los tipos espectrales de la tabla \ref{table:p6_color_int_UB} para las claves de limunosidad $V$, $IV$, $III$, $II$, $I_{b}$, $I_{a}$.

            Usar para $V$, $IV$, $III$ los $\left(B-V\right)$ de las clase de luminosidad $V$. Para los supergigantes $III$, $I_{b}$, $I_{a}$ usar los datos de la tabla \ref{table:p13_color_int_UB}.
    \end{problem}

    \begin{problem}\label{prob:14}
        Grafica los potenciales de ionizacion $\Xi$ de los elementos ``enrarecidos?'' como fuente para tipo espectral en función del logaritmo de la temperatura efectiva. Comentar.
    \end{problem}

    \begin{problem}\label{prob:15}
        \begin{ppart}\label{prob:15:B}
            Graficar $\log\left(\frac{L}{L_\odot}\right)$ vs $T_{BV}$ e interpretar, usando para ello el diagrama calculado en el ejecrcicio \ref{prob:4}.
        \end{ppart}
    \end{problem}

    \newpage
        \begin{table}
            \centering
            \caption{
                Curvas de sensibilidad de los filtros $U$, $B$ y $V$ del sistema fotométrico $UBV$, y del ojo humano para el día y la noche. \\ (Problema \ref{prob:1})
            }\label{table:p1_curvas_sensibilidad}
            \begin{tabularx}{\textwidth}{ *{6}{>{\Centering}X} }
                \hline
                $\lambda(\mu)$  & $U_\lambda$  & $B_\lambda$  & $V_\lambda$  & $O_{\textit{d\'ia}}$  & $O_{\textit{noche}}$
                %!-> BEGIN monkeypatch to manage spacings in table header
                \rule{0pt}{2.6ex}\rule[-1.2ex]{0pt}{0pt}\\
                & & & & & \\[-1.05em]\hline
                & & & & & \\[-1.05em]
                %!-> END monkeypatch to manage spacings in table header
                0.28  & 0.00  & -     & -     & -     & -    \\
                0.30  & 0.13  & -     & -     & -     & -    \\
                0.32  & 0.60  & -     & -     & -     & -    \\
                0.34  & 0.92  & -     & -     & -     & -    \\
                0.36  & 1.00  & 0.00  & -     & -     & -    \\
                0.38  & 0.72  & 0.13  & -     & -     & 0.00 \\
                0.40  & 0.09  & 0.92  & -     & -     & 0.02 \\
                0.42  & 0.00  & 1.00  & -     & 0.00  & 0.08 \\
                0.44  & -     & 0.92  & -     & 0.02  & 0.21 \\
                0.46  & -     & 0.76  & 0.00  & 0.06  & 0.41 \\
                0.48  & -     & 0.56  & 0.01  & 0.14  & 0.65 \\
                0.50  & -     & 0.39  & 0.36  & 0.32  & 0.90 \\
                0.52  & -     & 0.20  & 0.91  & 0.71  & 0.96 \\
                0.54  & -     & 0.07  & 0.98  & 0.95  & 0.68 \\
                0.56  & -     & 0.00  & 0.80  & 1.00  & 0.35 \\
                0.58  & -     & -     & 0.59  & 0.87  & 0.14 \\
                0.60  & -     & -     & 0.39  & 0.63  & 0.05 \\
                0.62  & -     & -     & 0.22  & 0.38  & 0.02 \\
                0.64  & -     & -     & 0.09  & 0.18  & 0.01 \\
                0.66  & -     & -     & 0.03  & 0.06  & 0.00 \\
                0.68  & -     & -     & 0.01  & 0.02  & -    \\
                0.70  & -     & -     & 0.00  & 0.00  & -    \\
                \hline
            \end{tabularx}
        \end{table}

        \begin{table}
            \centering
            \caption{
                Magnitudes Visuales Absolutas \\ (Problema \ref{prob:6})
            }\label{table:p6_cox_table2}
            \begin{tabularx}{\textwidth}{ *{7}{>{\Centering}X} }
                \hline
                T. Sp.  &  V  &  IV  &  III  &  II  &  Ib  &  Ia
                %!-> BEGIN monkeypatch to manage spacings in table header
                \rule{0pt}{2.6ex}\rule[-1.2ex]{0pt}{0pt}\\
                & & & & & & \\[-1.05em]\hline
                & & & & & & \\[-1.05em]
                %!-> END monkeypatch to manage spacings in table header
                O9  &  -4.8   &  -5.4  &  -6.0  &  -     &  -     &  -    \\
                B0  &  -4.1   &  -4.6  &  -5.0  &  -5.6  &  -6.2  &  -7.0 \\
                B1  &  -3.5   &  -3.9  &  -4.4  &  -5.1  &  -6.0  &  -7.0 \\
                B2  &  -2.5   &  -3.0  &  -3.6  &  -4.4  &  -5.9  &  -7.0 \\
                B3  &  -1.7   &  -2.3  &  -2.9  &  -3.9  &  -5.8  &  -7.0 \\
                B5  &  -1.1   &  -1.6  &  -2.2  &  -3.7  &  -5.7  &  -7.0 \\
                B7  &  -0.6   &  -1.0  &  -1.6  &  -3.6  &  -5.6  &  -7.0 \\
                B8  &  -0.2   &  -0.6  &  -1.2  &  -3.4  &  -5.5  &  -7.0 \\
                B9  &  +0.2   &  -0.3  &  -0.8  &  -3.1  &  -5.4  &  -7.0 \\
                A0  &  +0.6   &  +0.0  &  -0.6  &  -2.8  &  -4.9  &  -7.0 \\
                A1  &  +1.2   &  +0.3  &  -0.4  &  -2.6  &  -4.8  &  -7.0 \\
                A3  &  +1.7   &  +0.9  &  +0.0  &  -2.3  &  -4.6  &  -7.0 \\
                A5  &  +2.1   &  +1.2  &  +0.3  &  -2.1  &  -4.5  &  -7.0 \\
                A7  &  +2.4   &  +1.5  &  +0.5  &  -2.0  &  -4.5  &  -7.0 \\
                F0  &  +2.6   &  +1.7  &  +0.6  &  -2.0  &  -4.5  &  -7.0 \\
                F2  &  +3.0   &  +1.9  &  +0.6  &  -2.0  &  -4.5  &  -7.0 \\
                F5  &  +3.4   &  +2.1  &  +0.7  &  -2.0  &  -4.5  &  -7.0 \\
                F6  &  +3.7   &  +2.2  &  +0.7  &  -2.0  &  -4.5  &  -7.0 \\
                F8  &  +4.0   &  +2.4  &  +0.6  &  -2.0  &  -4.5  &  -7.0 \\
                G0  &  +4.4   &  +2.8  &  +0.6  &  -2.0  &  -4.5  &  -7.0 \\
                G2  &  +4.7   &  +3.0  &  +0.4  &  -2.1  &  -4.5  &  -7.0 \\
                G5  &  +5.2   &  +3.2  &  +0.3  &  -2.1  &  -4.5  &  -7.0 \\
                G8  &  +5.6   &  +3.2  &  +0.3  &  -2.1  &  -4.5  &  -7.0 \\
                K0  &  +5.9   &  +3.2  &  +0.2  &  -2.1  &  -4.5  &  -7.0 \\
                K2  &  +6.3   &  -     &  -0.1  &  -2.2  &  -4.5  &  -7.0 \\
                K3  &  +6.9   &  -     &  -0.2  &  -2.3  &  -4.5  &  -7.0 \\
                K5  &  +8.0   &  -     &  -0.3  &  -2.3  &  -4.5  &  -7.0 \\
                M0  &  +9.2   &  -     &  -0.4  &  -2.4  &  -4.5  &  -7.0 \\
                M1  &  +9.7   &  -     &  -0.5  &  -2.4  &  -4.5  &  -7.0 \\
                M2  &  +10.1  &  -     &  -0.5  &  -2.4  &  -4.5  &  -7.0 \\
                M3  &  +10.6  &  -     &  -0.5  &  -2.4  &  -4.5  &  -    \\
                M4  &  +11.3  &  -     &  -0.5  &  -2.4  &  -4.5  &  -    \\
                M5  &  +12.3  &  -     &  -     &  -     &  -     &  -    \\
                M6  &  +13.4  &  -     &  -     &  -     &  -     &  -    \\
                \hline
                \multicolumn{7}{l}{\footnotesize *Jhon P. Cox, R. Thomas Giuli, \emph{Stellar Structure, Physical}}\\
                \multicolumn{7}{l}{\footnotesize \emph{Principles}, p. 10)}
            \end{tabularx}
        \end{table}

        \begin{table}
            \centering
            \caption{
                Temperatura efectiva para tipos MK \\ (Problema \ref{prob:6})
            }\label{table:p6_cox_table3}
            \begin{tabularx}{\textwidth}{ *{7}{>{\Centering}X} }
                \hline
                T. Sp.  &  \multicolumn{6}{c}{$T_{eff}(K)$}
                %!-> BEGIN monkeypatch to manage spacings in table header
                \rule{0pt}{2.0ex}\rule[-1.0ex]{0pt}{0pt}\\
                & & & & & & \\[-1.1em]\hline
                & & & & & & \\[-1.1em]
                %!-> END monkeypatch to manage spacings in table header
                B0    & \multicolumn{6}{c}{27,000} \\
                B1    & \multicolumn{6}{c}{23,000} \\
                B2    & \multicolumn{6}{c}{20,000} \\
                B3    & \multicolumn{6}{c}{18,000} \\
                B5    & \multicolumn{6}{c}{16,000} \\
                B6.5  & \multicolumn{6}{c}{14,000} \\
                B8    & \multicolumn{6}{c}{12,500} \\
                B9    & \multicolumn{6}{c}{11,200} \\
                A0    & \multicolumn{6}{c}{10,400} \\
                A1    & \multicolumn{6}{c}{ 9,700} \\
                A2    & \multicolumn{6}{c}{ 9,100} \\
                A3    & \multicolumn{6}{c}{ 8,500} \\
                A5    & \multicolumn{6}{c}{ 8,200} \\
                A7    & \multicolumn{6}{c}{ 7,600} \\
                F0    & \multicolumn{6}{c}{ 7,200} \\[0.25em]
                \hline
                    &    V   &    IV  &  III   &  II    &  Ib    & Ia \\\hline
                F2  &  6900  &  6830  &  6800  &  6700  &  6600  &  \\
                F5  &  6700  &  6600  &  6500  &  6350  &  6200  &  \\
                F6  &  6500  &  6370  &  6250  &  6020  &  5800  &  \\
                F8  &  6200  &  6050  &  5900  &  5720  &  5450  &  \\
                G0  &  6000  &  5720  &  5500  &  5350  &  5050  &  \\
                G2  &  5740  &  5420  &  5100  &  4950  &  4750  &  \\
                G5  &  5520  &  5150  &  4800  &  4650  &  4500  &  \\
                G8  &  5320  &  4950  &  4600  &  4450  &  4300  &  \\
                K0  &  5120  &  4750  &  4400  &  4350  &  4100  &  \\
                K1  &  4920  &  4550  &  4150  &  4000  &  3850  &  \\
                K2  &  4760  &  4400  &  3970  &  3860  &  3750  &  \\
                K3  &  4600  &  -     &  3820  &  3720  &  3600  &  \\
                K5  &  4350  &  -     &  3700  &  3600  &  3500  &  \\
                K6  &  4000  &  -     &  -     &  -     &  -     &  \\
                M0  &  3750  &  -     &  3500  &  3400  &  3300  &  \\
                M1  &  3600  &  -     &  3300  &  3150  &  3050  &  \\
                M2  &  3350  &  -     &  3100  &  2050  &  -     &  \\
                M3  &  3100  &  -     &  2900  &  -     &  -     &  \\
                M4  &  -     &  -     &  2700  &  -     &  -     &  \\
                \hline
                \multicolumn{7}{l}{\footnotesize *Jhon P. Cox, R. Thomas Giuli, \emph{Stellar Structure, Physical Principles}, p. 11)}
            \end{tabularx}
        \end{table}

        \begin{table}
            \centering
            \caption{
                Colores intrínsecos y correción bolométrica \\ (Problema \ref{prob:6})
            }\label{table:p6_cox_table4}
            \begin{tabularx}{\textwidth}{ *{6}{>{\Centering}X} }
                \hline
                \multirow{3}{*}{T. Sp.} & \multicolumn{3}{c}{$B-V$}  & \multicolumn{2}{c}{$B.C.$}
                %!-> BEGIN monkeypatch to manage spacings in table header
                \rule{0pt}{2.0ex}\rule[-1.0ex]{0pt}{0pt}\\
                & & & & & \\[-1.1em]\cline{2-6}
                & & & & & \\[-1.1em]
                %!-> END monkeypatch to manage spacings in table header
                & \multicolumn{3}{c}{Clase de Luminosidad}  & \multicolumn{2}{c}{Classe de Luminosidad}
                %!-> BEGIN monkeypatch to manage spacings in table header
                \rule{0pt}{2.0ex}\rule[-1.0ex]{0pt}{0pt}\\
                & & & & & \\[-1.1em]\cline{2-6}
                & & & & & \\[-1.1em]
                %!-> END monkeypatch to manage spacings in table header
                & $V$ & $III$ & $I$ &$V$ & $III$ \\\hline
                05   & (-0.32) & (-0.32) & (-0.32) &  [-4.31] & \\
                09   & (-0.31) & (-0.31) & (-0.28) &  -3.34   & \\
                09.5 & (-0.30) & (-0.30) & (-0.27) &  [-3.68] & \\\\[-0.70em]
                B0   & (-0.30) & (-0.30) & (-0.24) & -3.17 & \\
                B1   & (-0.26) & (-0.26) & (-0.19) & -2.50 & \\
                B2   & (-0.24) & (-0.24) & (-0.17) & -2.23 & \\
                B3   & (-0.20) & (-0.20) & (-0.13) & -1.77 & \\
                B5   & (-0.16) & (-0.16) & (-0.09) & -1.39 & \\
                B6   & (-0.14) & (-0.14) & (-0.07) & -1.21 & \\
                B7   & (-0.12) & (-0.12) & (-0.05) & -1.04 & \\
                B8   & (-0.09) & (-0.09) & (-0.02) & -0.85 & \\
                B9   & (-0.06) & (-0.06) & ( 0.00) & -0.66 & \\\\[-0.70em]
                A0   & ( 0.00) & ( 0.00) & (+0.01) & -0.40 & \\
                A1   & (+0.03) & (+0.03) & (+0.01) & -0.32 & \\
                A2   & (+0.06) & (+0.06) & ( 0.00) & -0.25 & \\
                A3   & (+0.09) & -       & ( 0.00) & -0.20 & \\
                A5   & (+0.15) & (+0.15) & (+0.07) & -0.15 & \\
                A7   & (+0.20) & -       & (+0.13) & -0.12 & \\\\[-0.70em]
                F0   & (+0.30) & -       & (+0.24) & -0.08 & \\
                F2   & (+0.38) & -       & (+0.34) & -0.06 & \\
                F5   & (+0.45) & -       & (+0.45) & -0.04 & \\
                F6   & +0.47   & [+0.48] & -       & -0.04 & \\
                F7   & +0.50   & -       & -       & -0.04 & \\
                F8   & +0.53   & -       & [+0.68] & -0.05 & \\\\[-0.70em]
                G0   & +0.60   & -       & [+0.83] & -0.06 & \\
                G2   & +0.64   & -       & -       & -0.07 & \\
                G4   & -       & -       & [+0.97] & -     & \\
                G5   & +0.68   & +0.86   & -       & -0.10 & \\
                G8   & +0.72   & +0.93   & -       & -0.15 & \\
                \hline
                \multicolumn{6}{l}{\footnotesize (*) Jhon P. Cox, R. Thomas Giuli, \emph{Stellar Structure, Physical Principles}, pp. 12-13)}\\
                \multicolumn{6}{l}{\footnotesize (**) Continúa en la pág. siguiente}
            \end{tabularx}
        \end{table}

        \begin{table}
            \centering
            \begin{tabularx}{\textwidth}{ *{6}{>{\Centering}X} }
                \multicolumn{6}{l}{\footnotesize Continuación de la tabla \ref{table:p6_cox_table4}}\\
                \hline
                \multirow{3}{*}{T. Sp.} & \multicolumn{3}{c}{$B-V$}  & \multicolumn{2}{c}{$B.C.$}
                %!-> BEGIN monkeypatch to manage spacings in table header
                \rule{0pt}{1.5ex}\rule[-0.5ex]{0pt}{0pt}\\
                & & & & & \\[-1.05em]\cline{2-6}
                & & & & & \\[-1.05em]
                %!-> END monkeypatch to manage spacings in table header
                & \multicolumn{3}{c}{Clase de Luminosidad}  & \multicolumn{2}{c}{Classe de Luminosidad}
                %!-> BEGIN monkeypatch to manage spacings in table header
                \rule{0pt}{2.0ex}\rule[-1.0ex]{0pt}{0pt}\\
                & & & & & \\[-1.05em]\cline{2-6}
                & & & & & \\[-1.05em]
                %!-> END monkeypatch to manage spacings in table header
                & $V$ & $III$ & $I$ &$V$ & $III$
                %!-> BEGIN monkeypatch to manage spacings in table header
                \rule{0pt}{2.0ex}\rule[-1.0ex]{0pt}{0pt}\\
                & & & & & \\[-1.05em]\hline
                & & & & & \\[-1.05em]
                %!-> END monkeypatch to manage spacings in table header
                K0   & +0.81   & +1.01   & -       & -0.19 & \\
                K2   & +0.92   & +1.16   & [+1.37] & -0.25 & \\
                K3   & +0.98   & +1.29   & -       & -0.35 & \\
                K4   & -       & +1.40   & -       & -     & \\
                K5   & +1.18   & +1.52   & [+1.45] & -0.71 & \\
                K7   & +1.38   & -       & -       & -1.02 & \\\\[-0.65em]
                M0   & -      &  -  &  -       & -       & \\
                M1   & +1.48  &  -  &  -       & [-1.70] & \\
                M2   & -      &  -  &  [+1.67] & [-2.03] & \\
                M3   & +1.49  &  -  &  -       & [-2.35] & \\
                M4   & -      &  -  &  -       & [-2.7]  & \\
                M5   & +1.69  &  -  &  -       & [-3.1]  & \\
                M6   & -      &  -  &  -       & -       & \\
                \hline
                \multicolumn{6}{l}{\footnotesize (*) Jhon P. Cox, R. Thomas Giuli, \emph{Stellar Structure, Physical Principles}, pp. 12-13)}\\
                \multicolumn{6}{l}{\footnotesize (**) Continúa en la pág. siguiente}
            \end{tabularx}
        \end{table}

        \begin{table}
            \centering
            \caption{
                Estándar de la secuencia principal de edad cero \\ (Problema \ref{prob:6})
            }\label{table:p6_cox_table5}
            \begin{tabularx}{\textwidth}{ *{4}{>{\Centering}X} }
                \hline
                B-V    &  Mv     &  U-B    &  Mv
                %!-> BEGIN monkeypatch to manage spacings in table header
                \rule{0pt}{2.6ex}\rule[-1.2ex]{0pt}{0pt}\\
                & \\[-1.05em]\hline
                & \\[-1.05em]
                %!-> END monkeypatch to manage spacings in table header
                -0.25  &  -2.10  &  -0.90  &  -1.98  \\
                -0.20  &  -1.10  &  -0.80  &  -1.50  \\
                -0.15  &  -0.30  &  -0.70  &  -1.03  \\
                -0.10  &  +0.50  &  -0.60  &  -0.59  \\
                -0.05  &  +1.10  &  -0.50  &  -0.13  \\
                0.00   &  +1.50  &  -0.40  &  +0.27  \\
                +0.05  &  +1.74  &  -0.30  &  +0.66  \\
                +0.10  &  +2.00  &  -0.20  &  +1.02  \\
                +0.20  &  +2.45  &  -0.10  &  +1.30  \\
                +0.30  &  +2.95  &  0.00   &  +1.50  \\
                +0.40  &  +3.56  &         &         \\
                +0.50  &  +4.23  &         &         \\
                +0.60  &  +4.79  &         &         \\
                +0.70  &  +5.38  &         &         \\
                +0.80  &  +5.88  &         &         \\
                +0.90  &  +6.32  &         &         \\
                +1.00  &  +6.78  &         &         \\
                +1.10  &  +7.20  &         &         \\
                +1.20  &  +7.66  &         &         \\
                +1.30  &  +8.11  &         &         \\
                \hline
                \multicolumn{4}{l}{\footnotesize (*) Jhon P. Cox, R. Thomas Giuli, \emph{Stellar Structure, Physical Principles}, p. 13}
            \end{tabularx}
        \end{table}

        \begin{table}
            \centering
            \begin{tabularx}{0.5\textwidth}{ *{2}{>{\Centering}X} }
                \hline
                T. Sp.  & $\left(U-B\right)_{o}$
                %!-> BEGIN monkeypatch to manage spacings in table header
                \rule{0pt}{2.6ex}\rule[-1.2ex]{0pt}{0pt}\\
                & \\[-1.05em]\hline
                & \\[-1.05em]
                %!-> END monkeypatch to manage spacings in table header
                BO V  & -1.06 \\
                B5 V  & -0.55 \\
                A0 V  & -0.02 \\
                A5 V  &  0.10 \\
                F0 V  &  0.07 \\
                F5 V  &  0.03 \\
                G0 V  &  0.05 \\
                G5 V  &  0.19 \\
                K0 V  &  0.47 \\
                K5 V  &  1.10 \\
                M0 V  &  1.28 \\
                \hline
            \end{tabularx}
            \caption{
                Colores intrínsecos $\left(U-B\right)_O$
            }\label{table:p6_color_int_UB}
        \end{table}

        \begin{table}
            \centering
            \begin{tabularx}{0.5\textwidth}{ *{2}{>{\Centering}X} }
                \hline
                T. Sp.  & $\left(B-V\right)_{o}$
                %!-> BEGIN monkeypatch to manage spacings in table header
                \rule{0pt}{2.6ex}\rule[-1.2ex]{0pt}{0pt}\\
                & \\[-1.05em]\hline
                & \\[-1.05em]
                %!-> END monkeypatch to manage spacings in table header
                BO & -0.25 \\
                A0 &  0.00 \\
                F0 &  0.25 \\
                G0 &  0.70 \\
                G5 &  1.06 \\
                K0 &  1.39 \\
                K5 &  1.70 \\
                MO &  1.94 \\
                \hline
            \end{tabularx}
            \caption{
                Colores intrínsecos $\left(U-B\right)_O$
            }\label{table:p13_color_int_UB}
        \end{table}
\end{document}
