%% Preámbulo
%% ----------------------------------------------------------------------------
\documentclass[10pt,spanish,a4paper]{practice}
\usepackage[spanish]{babel}
\usepackage[utf8]{inputenc}
\usepackage[T1]{fontenc}
\usepackage{enumitem}
\usepackage{hyperref}
\usepackage{luximono}
\usepackage{textcomp}
\usepackage{amstext}
\usepackage{amsmath}
\usepackage{caption}
\usepackage{charter}

%% Settings
%% ----------------------------------------------------------------------------
% hyperref
\hypersetup{
    pdftitle={Física de Plasmas: Atm\'{o}sferas Estelares - Examen Final},
    pdfauthor={Mart\'{i}n Josemar\'{i}a Vuelta Rojas},
    pdfpagelayout=OneColumn,
    pdfnewwindow=true,
    pdfdisplaydoctitle=true,
    pdfstartview=XYZ,
    plainpages=false,
    unicode=true,
    bookmarksnumbered=true,
    bookmarksopen=true,
    bookmarksopenlevel=3,
    breaklinks=true,
    colorlinks=true,
    pdfborder={0 0 0}
}

% graphicx
\graphicspath{{resources/img/}}

% caption
\captionsetup{
    labelfont=bf,
    textfont=it,
    justification=centering,
    width=0.9\textwidth,
}

%% Definicion de comandos
%% ----------------------------------------------------------------------------
\makeatletter

%% Fuente de ancho fijo
\renewcommand{\ttdefault}{lmtt}
\renewcommand{\spanishtablename}{Tabla}

\makeatother

%% Documento
%% ----------------------------------------------------------------------------
\begin{document}
    %% Titulo, autor y resumen --------------------------------------------------
    \logo{unmsm.png}
    \university{
        Universidad Nacional Mayor de San Marcos\\
        {
            \scriptsize{
                \textit{
                    \textup{
                        Universidad del Perú, Decana de América}
                    }
            }
        }\\
        {\medskip}
        {
            \footnotesize {Facultad de Ciencias Físicas}
        }
    }
    \course{Física de Plasmas: Atmósferas Estelares}
    \title{Examen Final}
    \maketitle

    \begin{problem}
        Suponiendo que el campo de radiación de una atmósfera no gris se descompone en dos corrientes, una saliente $I_{\nu}^{+}\left(\tau_{\nu}\right)$ para $\mu > 0$ y una entrante $I_{\nu}^{-}\left(\tau_{\nu}\right)$ para $\mu < 0$.

        $$
            \pm\mu\frac{d I_{\nu}^{\pm}\left(\tau_{\nu}\right)}{d \tau_{\nu}} = I_{\nu}^{\pm}\left(\tau_{\nu}\right) - S_{\nu}
        $$

        Obtener la ecuación diferencial de segundo orden que permita calcular la Intensidad media en función de la opacidad y la función fuente.
    \end{problem}

    \begin{problem}
        Si la intensidad específica saliente de la atmósfera se  expresa de la siguiente manera

        $$
            I\left(0, \mu\right) = A + B \mu + C \mu^2.
        $$

        Empleando la relación de Eddingtong-Barbier determine la forma de la funcion fuente $S=S\left(\tau\right)$.
    \end{problem}

    \begin{problem}
        En una atmósfera gris y en equilibrio radiativo, en la que se suponela función fuente está dada por la función de Planck $S_\nu = B_\nu\left(T\right)$. Si la intensidad media del campo de radiación se expresa como

        $$
            J_{\nu} = \left[1 - \frac{1}{2}E_{2}\left(\tau\right)\right]B_{\nu}\left(\tau=\tfrac{2}{3}\right)
        $$

        \begin{problempart}
            Determine la temperatura $T(0)$ en función de la temperatura $T(\tau=\frac{2}{3})$.
        \end{problempart}

        \begin{problempart}
            ¿A qué $T(\tau=\frac{2}{3})$?
        \end{problempart}

        \begin{problempart}
            Empleando la relación de Barbier, determine el flujo astrfísico integrado en todas las longitudes de onda.
        \end{problempart}
    \end{problem}

    \begin{problem}
        Derive la ecuación

        $$
            J\left(\tau\right) = \frac{3}{4}F\left(\tau + Q + \sum_{\alpha=1}^{n-1}L_{\alpha}\mathrm{e}^{-k_{\alpha}\tau}\right)
        $$
    \end{problem}
\end{document}
