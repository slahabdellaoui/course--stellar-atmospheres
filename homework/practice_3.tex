%% Preámbulo
%% ----------------------------------------------------------------------------
\documentclass[12pt,a4paper]{practice}
\usepackage[spanish]{babel}
\usepackage[utf8]{inputenc}
\usepackage[T1]{fontenc}
\usepackage{enumitem}
\usepackage{hyperref}
\usepackage{luximono}
\usepackage{textcomp}
\usepackage{graphicx}
\usepackage{multirow}
\usepackage{tabularx}
\usepackage{ragged2e}
\usepackage{amssymb}
\usepackage{amstext}
\usepackage{caption}
\usepackage{charter}
\usepackage{mathabx}

%% Settings
%% ----------------------------------------------------------------------------
% hyperref
\hypersetup{
    pdftitle={Atm\'{o}sferas Estelares - Pr\'{a}ctica 1},
    pdfauthor={Mart\'{i}n Josemar\'{i}a Vuelta Rojas},
    pdfpagelayout=OneColumn,
    pdfnewwindow=true,
    pdfdisplaydoctitle=true,
    pdfstartview=XYZ,
    plainpages=false,
    unicode=true,
    bookmarksnumbered=true,
    bookmarksopen=true,
    bookmarksopenlevel=3,
    breaklinks=true,
    colorlinks=true,
    pdfborder={0 0 0}
}

% graphicx
\graphicspath{{resources/img/}}

% caption
\captionsetup{
    labelfont=bf,
    textfont=it,
    justification=centering,
    width=0.9\textwidth,
}

%% Definicion de comandos
%% ----------------------------------------------------------------------------
\makeatletter

%% Fuente de ancho fijo
\renewcommand{\ttdefault}{lmtt}
\renewcommand{\spanishtablename}{Tabla}

\makeatother

%% Documento
%% ----------------------------------------------------------------------------
\begin{document}
    %% Titulo, autor y resumen --------------------------------------------------
    \logo{unmsm.png}
    \university{
        Universidad Nacional Mayor de San Marcos\\
        {
            \scriptsize{
                \textit{
                    \textup{
                        Universidad del Perú, Decana de América}
                    }
            }
        }
    }
    \course{Atmósferas Estelares}
    \title{Práctica N\textdegree\ 3}b
    \maketitle


    \begin{problem}\label{prob:1}
    Para las estrellas de la \ref{tabla 1 }
    \end{problem}

        Calcular
            \begin{ppart}\label{prob:1:a}
            El indice $\Phi$ de cada estrella.
            \end{ppart}

            \begin{ppart}\label{prob:1:b}
            El exceso de valor $E(B-V)_{\Phi}$ usando el índice $\Phi$ obtenido en \ref{prob:1:a}
            \end{ppart}

            \begin{ppart}\label{prob:1:c}
            El exceso de valor $E(B-V)_{VBV}$ usando los colores intriusicos $(B-V)_{o}$ de la tabla adjuntada de la practica N°1.
            \end{ppart}

            \begin{ppart}\label{prob:1:d}
            El exceso de valor $E(B-V)_{2200}$ usando la depresión en $\lambda 2200$  NO SE QUE DICE primero un valor aroximado de $E(B-V)$ con la relación.
            $$
            E(B-V)^{ef} = 0.328\  \Delta ^{ef}\ 2200 + 0.040
            $$
            Corregir luego este valor con
            $$
            E(B-V)_{2200} = 1.189\  E(B-V)^{ef}- 0.025
            $$

            El índice $\Delta _{2200}$ esta definido por
             $$
             \Delta _{2200} = +2.5 \log \left(\bar{f}_{2200}/f_{2200}^{ob}\right)

             $$

            \end{ppart}

            \begin{ppart}\label{prob:1:e}
            Determinar el exceso de valo $E(B-V)_{TD-1}$, rectificando la distribución de energía de cada estrella. Para la rectificación usa la ley $A_{\lambda}E(B-V)$ de Sange y Methi, $(1979)$ dado en la \ref{table:p2} y una representación logaritmica de la distribución de flujos
            $$
            \log f_{\lambda} = a \cdot \log \lambda + b
            $$
            \end{ppart}

            \begin{ppart}\label{prob:1:f}
            Adaptar para cada estrella de la \ref{table:p1} el exceso $E(B-V)$ que resulta del promedio de las determinacines obtenidas en \ref{prob:1:c} \ref{prob:1:d} \ref{prob:1:e}.
            \end{ppart}

            \begin{ppart}\label{prob:1:g}
            Graficar el $E(B-V)$ adaptado con el $E(B-V)_{UBV}$
            \end{ppart}

            \begin{ppart}\label{prob:1:h}
            Graficar en un  mismo diagrama
            $$
            \log f_{\lambda} ^{ols} , \  a \log f_{\lambda} ^{com}
            $$
            en función de $\lambda$.
            \end{ppart}
        \end{problem}

        \begin{problem}\label{prob:2}
            \begin{ppart}\label{prob:2:a}
            Usando los flujos alternativos
            $$
            f_{\lambda} (\log/cm^{2}\cdot seg^{c} )
            $$
            de las estrellas del \ref{prob:1}, obtenidos por el satélite TD-1, y los índices de color intresecos $I(\lambda, V)$ de la \ref{table:p3}, determinar la ley de la absorcion interestelar $A{\lambda}/E(B-V)$ en el U-V lejano de esas dos estrellas.

            Usar
            $$
            R= A_{v}/E(B-V)=3.1
            $$
            El índice de color $I(\lambda, v)$ esta dedefinido por:
            $$
            I(\lambda, v)= -2.5 \log(f_{x}/f_{v})
            $$
            $f_{\lambda}$ : Flujo observado por TD-1
            $f_{v}$ : Flujo en $\lambda 5556 A_^{o}$
            $f_{v}  : f(v=0) \odot 10^{0.4}$
            $f_{v=0}  : 3.63*10^{-9} (mg/cm^{2}\odot reg A_^{o})$
            \end{ppart}



    \begin{table}
        \centering
            \caption{
                Tabla 1\\ (Problema \ref{prob:1})
            }\label{table:p1}
            \begin{tabularx}{\textwidth}{ *{6}{>{\Centering}X} }
                \hline
                $\l$  & $T.Sep$  & $V$  & $VB-V$  & $U.B$
                %!-> BEGIN monkeypatch to manage spacings in table header
                \rule{0pt}{2.6ex}\rule[-1.2ex]{0pt}{0pt}\\
                & & & & & \\[-1.05em]\hline
                & & & & & \\[-1.05em]
                %!-> END monkeypatch to manage spacings in table header
                47432  & 0.95 I   &  6.22 & 0.13  & -0.81    \\
                23478  & B3   IV  &  6.65 & 0.09  & -0.57    \\
                \end{tabularx}
        \end{table}




\end{document}
