%% Preámbulo
%% ----------------------------------------------------------------------------
\documentclass[10pt,spanish,a4paper,1p]{practice}
\usepackage[spanish]{babel}
\usepackage[utf8]{inputenc}
\usepackage[T1]{fontenc}
\usepackage{enumitem}
\usepackage{hyperref}
\usepackage{luximono}
\usepackage{textcomp}
\usepackage{amstext}
\usepackage{charter}

%% Settings
%% ----------------------------------------------------------------------------
% hyperref
\hypersetup{
  pdftitle={Atm\'{o}sferas Estelares - Pr\'{a}ctica 1},
  pdfauthor={Mart\'{i}n Josemar\'{i}a Vuelta Rojas},
  pdfpagelayout=OneColumn,
  pdfnewwindow=true,
  pdfdisplaydoctitle=true,
  pdfstartview=XYZ,
  plainpages=false,
  unicode=true,
  bookmarksnumbered=true,
  bookmarksopen=true,
  bookmarksopenlevel=3,
  breaklinks=true,
  colorlinks=true,
  pdfborder={0 0 0}
}

% graphicx
\graphicspath{{resources/img/}}

%% Definicion de comandos
%% ----------------------------------------------------------------------------
\makeatletter

%% Fuente de ancho fijo
\renewcommand{\ttdefault}{lmtt}
%\renewcommand{\familydefault}{\sfdefault}

\makeatother

%% Documento
%% ----------------------------------------------------------------------------
% \AtBeginDocument{\globalcolor{grey-900}}
\begin{document}
  %% Titulo, autor y resumen --------------------------------------------------
  \course{Atmósferas Estelares}
  \title{Práctica N\textdegree\ 1}
  \maketitle

  \begin{enumerate}[wide, labelwidth=!, labelindent=0pt, label=\textbf{\textrm{\arabic*)}}, ref=\arabic*]
    \item \label{prob:1} Dadas las curvas de sensibilidad de los filtros $U$, $B$ y $V$ del sistema fotométrico $UBV$, y las curvas de sensibilidad del ojo humano (día y noche); determine la \emph{longitud de onda equivalente} para cada una de las curvas de sensibilidad.

      \begin{table}[h!]
        \centering
        \begin{tabular}{ c | c | c | c | c | c }
          $\lambda(\mu)$ & $U_\lambda$ & $B_\lambda$ & $V_\lambda$ & $O_{\textit{d\'ia}}$ & $O_{noche}$ \\\hline
          0.28           & 0.00        &             &             &                   &      \\
          0.30           & 0.13        &             &             &                   &      \\
          0.32           & 0.60        &             &             &                   &      \\
          0.34           & 0.92        &             &             &                   &      \\
          0.36           & 1.00        & 0.00        &             &                   &      \\
          0.38           & 0.72        & 0.13        &             &                   & 0.00 \\
          0.40           & 0.09        & 0.92        &             &                   & 0.02 \\
          0.42           & 0.00        & 1.00        &             & 0.00              & 0.08 \\
          0.44           &             & 0.92        &             & 0.02              & 0.21 \\
          0.46           &             & 0.76        & 0.00        & 0.06              & 0.41 \\
          0.48           &             & 0.56        & 0.01        & 0.14              & 0.65 \\
          0.50           &             & 0.39        & 0.36        & 0.32              & 0.90 \\
          0.52           &             & 0.20        & 0.91        & 0.71              & 0.96 \\
          0.54           &             & 0.07        & 0.98        & 0.95              & 0.68 \\
          0.56           &             & 0.00        & 0.80        & 1.00              & 0.35 \\
          0.58           &             &             & 0.59        & 0.87              & 0.14 \\
          0.60           &             &             & 0.39        & 0.63              & 0.05 \\
          0.62           &             &             & 0.22        & 0.38              & 0.02 \\
          0.64           &             &             & 0.09        & 0.18              & 0.01 \\
          0.66           &             &             & 0.03        & 0.06              & 0.00 \\
          0.68           &             &             & 0.01        & 0.02              &      \\
          0.70           &             &             & 0.00        & 0.00              &      \\
          \noalign{\vskip 6pt}\hline
        \end{tabular}
      \end{table}

    \item \label{prob:2} Caclular la \emph{longitud de onda efectiva} del filtro $V$ para el flujo de un cuerpo negro con las siguientes temperaturas: $T = 25000 K$, $T = 10000 K$, $T = 5000 K$.

    \item \label{prob:3} Se tiene el flujo de cuerpo negro observado con un receptor cuya curva de sensibilidad es la curva de sensibilidad del filtro $V$ del sistema fotometrico $UBV$. Determine la longitud de onda del flujo monocromático efectivo para las temperaturas $T = 25000 K$, $10000 K$ y $5000 K$.

    \textbf{Recomendación}:
    Considerando que el flujo monocromático efectivo esta dado por

    $$\left\langle B\right\rangle = \frac{\displaystyle{\int_{0}^{\infty} V_{\lambda} B_{\lambda} \left(T\right) \mathrm{d}\lambda}}{\displaystyle{\int_{0}^{\infty} V_{\lambda} \mathrm{d}\lambda}}$$

    Determine para que valores de $\lambda$ se da la igualdad $\left\langle B\right\rangle = B_{\lambda} \left(T\right)$.

    %% ----------
    %% Enunciado original del problema 3
    %% ----------
    % Si se tiene un flujo de cuerpo negro observado con un receptor cuya curva de sensibilidad es la curva de sensibilidad del filtro $V$ del sistema fotometrico $UBV$. Dar la longitud de onda del flujo monocromático efectivo que se calcula con
    %
    % $$\left\langle B\right\rangle = \frac{\displaystyle{\int_{0}^{\infty} V_{\lambda} B_{\lambda} \left(T\right) \mathrm{d}\lambda}}{\displaystyle{\int_{0}^{\infty} V_{\lambda} \mathrm{d}\lambda}}$$
    %
    % (Asociar $\left\langle B\right\rangle = B_{\lambda} \left(T\right)$ y decir para que $\lambda$ se da la igualdad. Considerar $T = 25000 K$, $10000 K$ y $5000 K$)

    \item \label{prob:4} ¿Cuál es el cambio $\delta V$ en la magnitud $V$ del sistema fotmétrico $UBV$ que produce un cambio $\delta\lambda$ en la \emph{longitud de onda efectiva} calculada en \ref{prob:2}? Calcular $\delta\lambda = {\lambda}_{eq} - {\lambda}_{eff}$.

    \textbf{Recomendación}:
    Si $V = -2.5 \log f_V + C$ tomar $f_V \simeq B\left(T\right) $, $T=T\left({\lambda}_{eff}\right)$; suponer $B_{\lambda}\ \alpha\ {\lambda}^{-v} \mathrm{e}^{-\frac{hc}{\lambda k T}}$ ley de Wien; calcular $\left(\frac{\mathrm{d} \ln {f}_{\lambda}}{\mathrm{d}\lambda}\right)_{\lambda = {\lambda}_{eq}}$

    \item \label{prob:5} Cual es el cambio pocentual en $f_V$ que representa el cambio $\delta V$ calculado en \ref{prob:4}.

    \item \label{prob:6} Usando las tablas adjuntas y considerando los tipos espectrales \emph{O9},
    \emph{B0}, \emph{B2}, \emph{B5}, \emph{A0}, \emph{A5}, \emph{F0}, \emph{F5}, \emph{G0}, \emph{G5}, \emph{K0}, \emph{K5} y \emph{M0}.

      \begin{enumerate}
        \item Graficar $M_V$ vs. $\left(B-V\right)_{0}$ para cada clase de lumninosidad V, III y I.

        \item \label{prob:6:a} Calcular la temperatura de color $T_{BV}$ para los tipos espectrales de la secuencia principal.

        \item \label{prob:6:b} Usando la tabla siguiente de \emph{colores intrisicos} $\left(U-B\right)_O$, calcular las temperaturas de color $T_{UB}$.

        \item \label{prob:6:c} Comparar los tipos de colores $T_{UV}$ y $T_{VB}$ de los tipos espectrales dados en \ref{prob:6:b}.

          \begin{table}
            \centering
            \begin{tabular}{c | c}
              T. Sp. & $\left(U-B\rigth)_{o}$ \\\hline
              BO V   & -1.06 \\
              B5 V   & -0.55 \\
              A0 V   & -0.02 \\
              A5 V   &  0.10 \\
              F0 V   &  0.07 \\
              F5 V   &  0.03 \\
              G0 V   &  0.05 \\
              G5 V   &  0.19 \\
              K0 V   &  0.47 \\
              K5 V   &  1.10 \\
              M0 V   &  1.28
            \end{tabular}
          \end{table}

        \item\label{prob:6:d} Comparar las temperaturas de color $T_{BV}$ y $T_{UV}$ de los espectrales dados en \ref{prob:6:c} con las temperaturas espectrales de la tabla adjuntada.

      \begin{enumerate}
            \item\label{prob:6:d:i} Cuál de las tempraturas de color parece aproximarse mejor desde un punto de vista cuantitativo a la temperatura afectiva?\\
            \item\label{prob:6:d:ii} Cúal de las tempraturas de color parece aproximarse mejor desde un punto de vista cualitativo a la temperatura efectiva?
      \end{enumerate}
      \end{enumerate}

    \item\label{prob:7} Dibuja un diagrama $\left(V-B\right)_O$ VS $\left(B-V\right)_O$ pero los tipos espectrales del ejercicio \ref{prob:6:c}. Superponer en el mismo diagrama la relación color-color del cuerpo negro.
              El cuerpo negro ajusto bien las observaciones.

    \item\label{prob:8}
        \begin{enumerate}
        \item \label{prob:8:a} Para los tipos espectrales de la tabla en \ref{prob:6:c} Calcular el coeficiente Q.

        \item \label{prob:8:b} Calcular Q suponiendo $T_{BV}$ = $T_{UB}$ = $T_{EH}$

        \item \label{prob:8:c} Calcular Q usando sólo $T_{BV}$

        \item \label{prob:8:d} Calcular Q usando sólo $T_{UB}$
      \end{enumerate}

    \item\label{prob:9}Calcular un diagrama Log L9L0 vs Log $T_{EH}$ = para log $\left(RIRO\right)$ = -3, -2, -1, 0, +1, +2.

       \begin{enumerate}
       \item\label{prob:9:a} Mostrar que la correccion bolometricas enre magnitudes NO SE QUE DICE  es igual a la correccion bolometrica entre magnitudes absolutas.
       \end{enumerate}

    \item\label{prob:10} Discutir porque la correccion bolometrica es funcion proncipalmente de la temperatura efectiva y no del sodio.
    En esas condiciones la correccion bolometrica depende de la clase de luminosidad?¿Porque?

    \item\label{prob:11} Si log fx = a. log + b en un internolo $\left(NO SE QUE DICE\right)$ donde fx y el flujo monocromatico NO SE QUE DICE de una estrella, mientras que el gradiente de color en ese intermedio es $Q_{v1d2}$ = $\left(5+a\right)$.





\end{document}
