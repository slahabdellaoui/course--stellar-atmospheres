%% Preámbulo
%% ----------------------------------------------------------------------------
\documentclass[12pt,a4paper]{practice}
\usepackage[spanish]{babel}
\usepackage[utf8]{inputenc}
\usepackage[T1]{fontenc}
\usepackage{enumitem}
\usepackage{hyperref}
\usepackage{luximono}
\usepackage{textcomp}
\usepackage{graphicx}
\usepackage{amstext}
\usepackage{amsmath}
\usepackage{caption}
\usepackage{charter}

%% Settings
%% ----------------------------------------------------------------------------
% hyperref
\hypersetup{
    pdftitle={Atm\'{o}sferas Estelares - Pr\'{a}ctica 1},
    pdfauthor={Mart\'{i}n Josemar\'{i}a Vuelta Rojas},
    pdfpagelayout=OneColumn,
    pdfnewwindow=true,
    pdfdisplaydoctitle=true,
    pdfstartview=XYZ,
    plainpages=false,
    unicode=true,
    bookmarksnumbered=true,
    bookmarksopen=true,
    bookmarksopenlevel=3,
    breaklinks=true,
    colorlinks=true,
    pdfborder={0 0 0}
}

% graphicx
\graphicspath{{resources/img/}}

% caption
\captionsetup{
    labelfont=bf,
    textfont=it,
    justification=centering,
    width=0.9\textwidth,
}

%% Definicion de comandos
%% ----------------------------------------------------------------------------
\makeatletter

%% Fuente de ancho fijo
\renewcommand{\ttdefault}{lmtt}
\renewcommand{\spanishtablename}{Tabla}

%% Mathematical commnads
\newcommand\integrate[4]{%
    \int_{#3}^{#4} {#1 \mathrm{d} #2}
}
\makeatother

\newcommand\trapezerule[4]{%
    \sum_{n=#3}^{#4} {\frac{#1(#2_{n+1}) + #1(#2_n)}{2} \times (#2_{n+1} - #2_n)}
}

\makeatother

%% Documento
%% ----------------------------------------------------------------------------
\begin{document}
    %% Titulo, autor y resumen --------------------------------------------------
    \logo{unmsm.png}
    \university{
        Universidad Nacional Mayor de San Marcos\\
        {
            \scriptsize{
                \textit{
                    \textup{
                        Universidad del Perú, Decana de América}
                    }
            }
        }
    }
    \course{Atmósferas Estelares}
    \title{Práctica N\textdegree\ 1}
    \maketitle

    \begin{problem}\label{prob:1}
        Determine la \emph{longitud de onda equivalente} para cada una de las curvas de sensibilidad dadas en la tabla \ref{table:1}.

        \begin{table}[h!]
            \centering
            \begin{tabular}{ c | c | c | c | c | c }
                $\lambda(\mu m)$  & $U_\lambda$  & $B_\lambda$  & $V_\lambda$  & $O_{\textit{d\'ia}}$  & $O_{\textit{noche}}$ \\
                &  &  &  &  &  \\[-0.8em]\hline
                &  &  &  &  &  \\[-0.8em]
                0.28  & 0.00  &  &  &  &  \\
                0.30  & 0.13  &  &  &  &  \\
                0.32  & 0.60  &  &  &  &  \\
                0.34  & 0.92  &  &  &  &  \\
                0.36  & 1.00  & 0.00  &  &  &  \\
                0.38  & 0.72  & 0.13  &  &  & 0.00 \\
                0.40  & 0.09  & 0.92  &  &  & 0.02 \\
                0.42  & 0.00  & 1.00  &  & 0.00  & 0.08 \\
                0.44  &  & 0.92  &  & 0.02  & 0.21 \\
                0.46  &  & 0.76  & 0.00  & 0.06  & 0.41 \\
                0.48  &  & 0.56  & 0.01  & 0.14  & 0.65 \\
                0.50  &  & 0.39  & 0.36  & 0.32  & 0.90 \\
                0.52  &  & 0.20  & 0.91  & 0.71  & 0.96 \\
                0.54  &  & 0.07  & 0.98  & 0.95  & 0.68 \\
                0.56  &  & 0.00  & 0.80  & 1.00  & 0.35 \\
                0.58  &  &  & 0.59  & 0.87  & 0.14 \\
                0.60  &  &  & 0.39  & 0.63  & 0.05 \\
                0.62  &  &  & 0.22  & 0.38  & 0.02 \\
                0.64  &  &  & 0.09  & 0.18  & 0.01 \\
                0.66  &  &  & 0.03  & 0.06  & 0.00 \\
                0.68  &  &  & 0.01  & 0.02  &  \\
                0.70  &  &  & 0.00  & 0.00  &  \\
                \hline
            \end{tabular}
            \caption{
                Curvas de sensibilidad de los filtros $U$, $B$ y $V$ del sistema fotométrico $UBV$, y del ojo humano para el día y la noche.
            }\label{table:1}
        \end{table}

        \begin{solution}
            Para una curva de sensiblidad $s\left(\lambda \right)$, la longitud de onda equivalente $\lambda_{eq}$ viene dada por la fórmula

            \begin{equation}\label{eq:equivalent_wavelength}
                \lambda_{eq} = \frac{\displaystyle\integrate{\lambda s(\lambda)}{\lambda}{0}{\infty} }{\displaystyle\integrate{s(\lambda)}{\lambda}{0}{\infty}}.
            \end{equation}

            Con los datos proporcionados en  la tabla \ref{table:1}, calculamos las integrales en la ecuación \ref{eq:equivalent_wavelength} usando el método del trapecio

            \begin{equation*}
                \integrate{f(x)}{x}{a}{b} = \trapezerule{f}{x}{0}{N-1}\,,\,a=x_{0} \wedge b = x_N.
            \end{equation*}

            Los resultados obtenidos se muestran en la tabla siguiente:

            \begin{table}[h!]
                \centering
                \begin{tabular}{ l | c }
                      & $\lambda_{eq}\ (\mu m)$ \\
                      &  \\[-0.8em]\hline
                      &  \\[-0.8em]
                    $U_{\lambda}$  & 0.35 \\
                    $B_{\lambda}$  & 0.44 \\
                    $V_{\lambda}$  & 0.55 \\
                    $O_{\textrm{día}}$  & 0.56 \\
                    $O_{\textrm{noche}}$  & 0.51 \\
                    \hline
                \end{tabular}
            \end{table}
        \end{solution}
    \end{problem}

    \begin{problem}\label{prob:2}
        Empleando los datos de la tabla \ref{table:1}, caclular la \emph{longitud de onda efectiva} del filtro $V$ para el flujo de un cuerpo negro con las siguientes temperaturas: $T = 25000 K$, $T = 10000 K$, $T = 5000 K$.

        \begin{solution}
            Para una curva de sensiblidad $s\left(\lambda \right)$ y una fuente con irradiancia $f(T, \lambda)$, la longitud de onda effectiva $\lambda_{eff}$ viene dada por la fórmula

            \begin{equation}\label{eq:effective_wavelength}
                \lambda_{eff} = \frac{\displaystyle\integrate{\lambda f(T, \lambda) s(\lambda)}{\lambda}{0}{\infty} }{\displaystyle\integrate{f(T, \lambda) s(\lambda)}{\lambda}{0}{\infty}}.
            \end{equation}

            Para el cuerpo negro tenemos que la irradiancia está data por la funcion de Planck

            \begin{equation}\label{eq:effective_wavelength}
                f(T, \lambda) = \frac{2 h c^2}{\lambda^5}\frac{1}{\exp(\frac{h c}{\lambda k T}) - 1}.
            \end{equation}

            Empleando el método del trapecio nuevamente, tabulamos el valor de $\lambda_{eff}$ para las temperaturas $T = 25000 K$, $T = 10000 K$, $T = 5000 K$ para el filtro V en la tabla siguiente.

            \begin{table}[h!]
                \centering
                \begin{tabular}{ c | c }
                    $T\ (K)$  & $\lambda_{eff}\ (\mu m)$ \\
                      &  \\[-0.8em]\hline
                      &  \\[-0.8em]
                    25000  & 0.547 \\
                    10000  & 0.549 \\
                    5000   & 0.554 \\
                    \hline
                \end{tabular}
            \end{table}
        \end{solution}
    \end{problem}

    \begin{problem}\label{prob:3}
        Se tiene el flujo de cuerpo negro observado con un receptor cuya curva de sensibilidad es la curva de sensibilidad del filtro $V$ del sistema fotometrico $UBV$. Determine la longitud de onda del flujo monocromático efectivo para las temperaturas $T = 25000 K$, $10000 K$ y $5000 K$.

        \begin{recommendation}
            Considerando que el flujo monocromático efectivo está dado por

                $$\left\langle B\right\rangle = \frac{\displaystyle{\int_{0}^{\infty} V_{\lambda} B_{\lambda} \left(T\right) \mathrm{d}\lambda}}{\displaystyle{\int_{0}^{\infty} V_{\lambda} \mathrm{d}\lambda}},$$

            determine para que valores de $\lambda$ se da la igualdad $\left\langle B\right\rangle = B_{\lambda} \left(T\right)$.
        \end{recommendation}
    \end{problem}

    \begin{problem}\label{prob:4}
        ¿Cuál es el cambio $\delta V$ en la magnitud $V$ del sistema fotmétrico $UBV$ que produce un cambio $\delta\lambda$ en la \emph{longitud de onda efectiva} calculada en \ref{prob:2}?

        Calcular $\delta\lambda = {\lambda}_{eq} - {\lambda}_{eff}$.

        \begin{recommendation}
            Si $V = -2.5 \log f_V + C$ tomar $f_V \simeq B\left(T\right) $, $T=T\left({\lambda}_{eff}\right)$; suponer $B_{\lambda}\ \alpha\ {\lambda}^{-v} \mathrm{e}^{-\frac{hc}{\lambda k T}}$ (Ley de Wien); calcular $\left(\frac{\mathrm{d} \ln {f}_{\lambda}}{\mathrm{d}\lambda}\right)_{\lambda = {\lambda}_{eq}}$
        \end{recommendation}
    \end{problem}

    \begin{problem}\label{prob:5}
        ¿Cuál es el cambio pocentual en $f_V$ que representa el cambio $\delta V$ calculado en \ref{prob:4}?.
    \end{problem}

    \begin{problem}\label{prob:6}
        Usando las tablas adjuntas (Referencia del COX) y considerando los tipos espectrales \emph{O9}, \emph{B0}, \emph{B2}, \emph{B5}, \emph{A0}, \emph{A5}, \emph{F0}, \emph{F5}, \emph{G0}, \emph{G5}, \emph{K0}, \emph{K5} y \emph{M0}.

        \begin{problempart}\label{prob:6:a}
            Graficar $M_V$ vs. $\left(B-V\right)_{0}$ para cada clase de lumninosidad V, III y I.
        \end{problempart}

        \begin{problempart}\label{prob:6:b}
            Calcular la temperatura de color $T_{BV}$ para los tipos espectrales de la secuencia principal.
        \end{problempart}

        \begin{problempart}\label{prob:6:c}
            Usando los datos de la tabla \ref{table:2} de \emph{colores intrínsecos} $\left(U-B\right)_O$, calcular las temperaturas de color $T_{UB}$.
        \end{problempart}

        \begin{table}[h!]
            \centering
            \begin{tabular}{c | c}
                T. Sp.  & $\left(U-B\right)_{o}$ \\
                &  \\[-0.8em]\hline
                &  \\[-0.8em]
                BO V  & -1.06 \\
                B5 V  & -0.55 \\
                A0 V  & -0.02 \\
                A5 V  &  0.10 \\
                F0 V  &  0.07 \\
                F5 V  &  0.03 \\
                G0 V  &  0.05 \\
                G5 V  &  0.19 \\
                K0 V  &  0.47 \\
                K5 V  &  1.10 \\
                M0 V  &  1.28 \\
                \hline
            \end{tabular}
            \caption{
                Colores intrínsecos $\left(U-B\right)_O$
            }\label{table:2}
        \end{table}

        \begin{problempart}\label{prob:6:d}
            Comparar los tipos de colores $T_{UV}$ y $T_{VB}$ de los tipos espectrales dados en \ref{prob:6:c}.
        \end{problempart}

        \begin{problempart}\label{prob:6:e}
            Comparar las temperaturas de color $T_{BV}$ y $T_{UV}$ de los espectrales dados en \ref{prob:6:b} con las temperaturas espectrales de la tabla adjuntada.

            \begin{itemize}
                \item ¿Cuál de las tempraturas de color parece aproximarse mejor desde un punto de vista cuantitativo a la temperatura afectiva?
                \item ¿Cúal de las tempraturas de color parece aproximarse mejor desde un punto de vista cualitativo a la temperatura efectiva?
            \end{itemize}
        \end{problempart}
    \end{problem}

    \begin{problem}\label{prob:7}
        Dibuja un diagrama $\left(V-B\right)_O$ vs. $\left(B-V\right)_O$ pero los tipos espectrales del ejercicio \ref{prob:6:b}. Superponer en el mismo diagrama la relación color-color del cuerpo negro. El cuerpo negro ajusto bien las observaciones.
    \end{problem}

    \begin{problem}\label{prob:8}
        Para los tipos espectrales de la tabla \ref{table:2}

            \begin{problempart}\label{prob:8:a}
                Calcular el coeficiente $Q$.
            \end{problempart}

            \begin{problempart}\label{prob:8:b}
                Calcular Q suponiendo $T_{BV}$ = $T_{UB}$ = $T_{EH}$.
            \end{problempart}

            \begin{problempart}\label{prob:8:c}
                Calcular Q usando sólo $T_{BV}$.
            \end{problempart}

            \begin{problempart}\label{prob:8:d}
                Calcular Q usando sólo $T_{UB}$.
            \end{problempart}
    \end{problem}

    \begin{problem}\label{prob:9}
        Calcular un diagrama Log L9L0 vs Log $T_{EH}$ = para log $\left(RIRO\right)$ = -3, -2, -1, 0, +1, +2.

            \begin{problempart}\label{prob:9:a}
                Mostrar que la correccion bolometricas enre magnitudes NO SE QUE DICE  es igual a la correccion bolometrica entre magnitudes absolutas.
            \end{problempart}
    \end{problem}

    \begin{problem}\label{prob:10}
        Discutir porque la correccion bolometrica es funcion proncipalmente de la temperatura efectiva y no del sodio.

        En esas condiciones la correccion bolometrica depende de la clase de luminosidad?¿Porque?
    \end{problem}

    \begin{problem}\label{prob:11}
        Si $\log f_{\lambda} = a\log\lambda + b$ en un internolo $\left(NO SE QUE DICE\right)$ donde fx y el flujo monocromatico NO SE QUE DICE de una estrella, mientras que el gradiente de color en ese intermedio es $Q_{v1d2}$ = $\left(5+a\right)$.
    \end{problem}

    \begin{problem}\label{prob:12}
        \begin{problempart}\label{prob:12:a}
            Mostrar que el gradiente de color de una radiacion de cuerpo negro esta dado por
            $$
                \Phi = \frac{c_2}{T}\left(1 - \mathrm{e}^{-\frac{c_2}{\lambda T}}\right)
            $$

            $$
                c_2 = \frac{hc}{k} \approx 1.43883\ \mathrm{cm \cdot K}
            $$
        \end{problempart}

        \begin{problempart}\label{prob:12:b}
            ¿En que region $\Phi = \frac{c_2}{\lambda T}$?
        \end{problempart}

        \begin{problempart}\label{prob:12:c}
            Calcular $\Phi$ usando $\lambda = \frac{1}{2}\left(\lambda_B + \lambda_V\right)_{eq}$ y $T = T_{BV}$ para los tipos espectrales de \ref{prob:6:d}.
        \end{problempart}
    \end{problem}

    \begin{problem}\label{prob:13}
        Transofrmar
            \begin{itemize}
                \item $M_{V}$ en $M_{bol}$.
                \item $M_{V}$ en $T_{BV}$.
            \end{itemize}

            Para los tipos espectrales de la tabla \ref{table:2} para las claves de limunosidad $V$, $IV$, $III$, $II$, $I_{b}$, $I_{a}$.

            Usar para $V$, $IV$, $III$ los $\left(B-V\right)$ de las clase de luminosidad $V$. Para los supergigantes $III$, $I_{b}$, $I_{a}$ usar los datos de la tabla \ref{table:3}.
    \end{problem}

    \begin{problem}\label{prob:14}
        Grafica los potenciales de ionizacion $\Xi$ de los elementos ``enrarecidos?'' como fuente para tipo espectral en función del logaritmo de la temperatura efectiva. Comentar.
    \end{problem}

    \begin{problem}\label{prob:15}
        \begin{problempart}\label{prob:15:B}
            Graficar $\log\left(\frac{L}{L_\odot}\right)$ vs $T_{BV}$ e interpretar, usando para ello el diagrama calculado en el ejecrcicio \ref{prob:4}.
        \end{problempart}

        \begin{table}[h!]
            \centering
            \begin{tabular}{c | c}
                T. Sp.  & $\left(B-V\right)_{o}$ \\\\
                  &  \\[-0.8em]\hline
                  &  \\[-0.8em]
                BO & -0.25 \\
                A0 &  0.00 \\
                F0 &  0.25 \\
                G0 &  0.70 \\
                G5 &  1.06 \\
                K0 &  1.39 \\
                K5 &  1.70 \\
                MO &  1.94 \\
                \hline
            \end{tabular}
            \caption{
                Colores intrínsecos $\left(U-B\right)_O$
            }\label{table:3}
        \end{table}
    \end{problem}
\end{document}
