%% Preámbulo
%% ----------------------------------------------------------------------------
\documentclass[12pt,spanish,a4paper]{practice}
\usepackage[spanish]{babel}
\usepackage[utf8]{inputenc}
\usepackage[T1]{fontenc}
\usepackage{enumitem}
\usepackage{hyperref}
\usepackage{luximono}
\usepackage{textcomp}
\usepackage{amstext}
\usepackage{caption}
\usepackage{charter}

%% Settings
%% ----------------------------------------------------------------------------
% hyperref
\hypersetup{
    pdftitle={Atm\'{o}sferas Estelares - Pr\'{a}ctica 1},
    pdfauthor={Mart\'{i}n Josemar\'{i}a Vuelta Rojas},
    pdfpagelayout=OneColumn,
    pdfnewwindow=true,
    pdfdisplaydoctitle=true,
    pdfstartview=XYZ,
    plainpages=false,
    unicode=true,
    bookmarksnumbered=true,
    bookmarksopen=true,
    bookmarksopenlevel=3,
    breaklinks=true,
    colorlinks=true,
    pdfborder={0 0 0}
}

% graphicx
\graphicspath{{resources/img/}}

% caption
\captionsetup{
    labelfont=bf,
    textfont=it,
    justification=centering,
    width=0.9\textwidth,
}

%% Definicion de comandos
%% ----------------------------------------------------------------------------
\makeatletter

%% Fuente de ancho fijo
\renewcommand{\ttdefault}{lmtt}
\renewcommand{\spanishtablename}{Tabla}

\makeatother

%% Documento
%% ----------------------------------------------------------------------------
\begin{document}
    %% Titulo, autor y resumen --------------------------------------------------
    \logo{unmsm.png}
    \university{
        Universidad Nacional Mayor de San Marcos\\
        {
            \scriptsize{
                \textit{
                    \textup{
                        Universidad del Perú, Decana de América}
                    }
            }
        }
    }
    \course{Atmósferas Estelares}
    \title{Examen Parcial de Física de Plasmas}
    \maketitle

    \begin{problem}
        En un sistema binario se conocen la magnitud absoluta del sstema y la magnitud absoluta de una de las componentes. Dar la expresión que permite calcular la magnitud absoluta de la otra componente.
    \end{problem}

    \begin{problem}
        Observacionalmente se obtiene que para estrellas masivas de la secuencia principal existe la siguiente proporcionalidad entre la luminosidad bolométrica y la masa: $L\ \alpha\ M^{4}$. ¿A qué será proporcional la relación $T_{eff}/T$? Donde $T_{eff}$ es la temperatura efectiva y $T$ es la temperatura media.
    \end{problem}

    \begin{problem}
        Si $f_{\lambda}^{obs}$ es el flujo observado de una estrella en una longitud de onda dada y $f_{\lambda}^{0}$ la correccion sin absorción interestelar. Dar la relación entre ambas en función de la absorción interestelar.
    \end{problem}
    
    \begin{problem}
        si la intensidad específica no depende del ángulo es decir es isótropa, demostrar que es flujo radiativo es nulo $F_{V}=0$ y se cumple que $J_{V}=3K_{V}$.
    \end{problem}

    \begin{problem}
       Usando la relación de Schwarzchild-Milne
   
            $$
                J(\tau) = \frac{1}{2} \displaystyle{\int_{0}^{\infty}} S(\tau) E_{1} (t-\tau) dt
            $$

        y considerando

            $$
                S=J= \frac{3}{4} F\left(\tau+\frac{2}{3}\right)
            $$
        
        Demostrar que:

            $$
                J(\tau) = \frac{3}{4}F\left(\tau+\frac{2}{3}-\frac{1}{3}E_{2}(\tau)+\frac{1}{2}E_{3}(\tau)\right)
            $$
    \end{problem}
\end{document}
