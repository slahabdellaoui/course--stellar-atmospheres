%% Preámbulo
%% ----------------------------------------------------------------------------
\documentclass[12pt,spanish,a4paper]{practice}
\usepackage[spanish]{babel}
\usepackage[utf8]{inputenc}
\usepackage[T1]{fontenc}
\usepackage{enumitem}
\usepackage{hyperref}
\usepackage{luximono}
\usepackage{textcomp}
\usepackage{amsmath}
\usepackage{amstext}
\usepackage{caption}
\usepackage{charter}

%% Settings
%% ----------------------------------------------------------------------------
% hyperref
\hypersetup{
    pdftitle={Atm\'{o}sferas Estelares - Pr\'{a}ctica 1},
    pdfauthor={Mart\'{i}n Josemar\'{i}a Vuelta Rojas},
    pdfpagelayout=OneColumn,
    pdfnewwindow=true,
    pdfdisplaydoctitle=true,
    pdfstartview=XYZ,
    plainpages=false,
    unicode=true,
    bookmarksnumbered=true,
    bookmarksopen=true,
    bookmarksopenlevel=3,
    breaklinks=true,
    colorlinks=true,
    pdfborder={0 0 0}
}

% graphicx
\graphicspath{{resources/img/}}

% caption
\captionsetup{
    labelfont=bf,
    textfont=it,
    justification=centering,
    width=0.9\textwidth,
}

%% Definicion de comandos
%% ----------------------------------------------------------------------------
\makeatletter

%% Fuente de ancho fijo
\renewcommand{\ttdefault}{lmtt}
\renewcommand{\spanishtablename}{Tabla}

\makeatother

%% Documento
%% ----------------------------------------------------------------------------
\begin{document}
    %% Titulo, autor y resumen --------------------------------------------------
    \logo{unmsm.png}
    \university{
        Universidad Nacional Mayor de San Marcos\\
        {
            \scriptsize{
                \textit{
                    \textup{
                        Universidad del Perú, Decana de América}
                    }
            }
        }
    }
    \course{Atmósferas Estelares}
    \title{Práctica N\textdegree\ 5}
    \maketitle
    \begin{problem}\label{prob:1}
    Obtener el valor del factor de Eddington $f_{v} (J_{v}=f_{v}.K_{v})$ para los siguientes representaciones de la intensidad especifica:

        \begin{ppart}\label{prob:1:a}
            \begin{equation*}
                I_{\nu}(\tau_{\nu},\mu) = \left\{
                \begin{array}{l c l}
                    I_{\nu}^{o}(\tau_{\nu}) & ; & 0 \leq \mu \leq 1 \\
                    0 & ; &  \mu < 0
                \end{array}
                \right. 
            \end{equation*}
            
        \end{ppart}
        
        \begin{ppart}\label{prob:1:b}
            \begin{equation*}
                I_{\nu}(\tau_{\nu},\mu) = \left\{
                \begin{array}{l c l}
                    I_{\nu}^{o}(\tau_{\nu}) & ; & 0 \leq \mu \leq 1 \\
                    I_{\nu}^{1}(\tau_{\nu}) & ; & -1 \leq \mu \leq 0
                \end{array}
                \right. 
            \end{equation*}
            
        \end{ppart}

        \begin{ppart}\label{prob:1:c}
            \begin{equation*}
                I_{\nu}(\tau_{\nu},\mu) = C_{\nu}^{(0)}(\tau_{\nu}) + \sum_{k=1}^{\infty} C_{\nu (\tau_{\nu})}^{2k-1} . \mu^{2k-1}
            \end{equation*}
        \end{ppart}

        \begin{ppart}\label{prob:1:d}
            \begin{equation*}
                I_{\nu}(\tau_{\nu},\mu) = d_{\nu}^{(0)}(\tau_{\nu}) + \sum_{k=1}^{\infty} d_{\nu (\tau_{\nu})}^{2\mu} . \mu^{2k}
            \end{equation*}
        \end{ppart}

        \begin{ppart}\label{prob:1:e}
            \begin{equation*}
                I_{\nu}(\tau_{\nu},\mu) = I_{\nu}^{0}(\tau_{\nu}) . \delta(\mu - \mu_{0})
            \end{equation*}
        \end{ppart}
    \end{problem}

    \begin{problem}\label{prob:2}
        Para cada uno de los casos del ejercicio 1 obtener la expresión del flujo astrofisico $F_{\nu}(\tau_{\nu})$.
    \end{problem}
    
    \begin{problem}\label{prob:3}
        Obtener el factor de Eddington $f=f(\tau)$ usando las relacione de Schwarzschild-Milne y para los casos donde

        \begin{ppart}\label{prob:3:a}
                $S(\tau) = \frac{3}{4}F(\tau + \frac{2}{3})$
        \end{ppart}

        \begin{ppart}\label{prob:3:b}
            $S(\tau) = S$
        \end{ppart}

        
        Dar el noln numerico de f en los limites $\tau = 0$ y $\tau \to \infty$.
    \end{problem}
    
    \begin{problem}\label{prob:4}
        
        utilizando la aproxiomacion de Eddington $(f = 3)$ obtener una expresion de la intensidad media $J_{\nu}(\tau_{\nu})$ para el caso donde la funcion fuente esta dada por la expresión 

        $$S_{\nu}(\tau_{nu}) = S_{\nu}^{(0)} + s_{\nu}^{1}.\tau_{\nu} + S_{\nu}^{(2)}.e^{-\tau_{\nu} / \tau_{0}}$$

    \end{problem}
    
    \begin{problem}\label{prob:5}

        \begin{ppart}\label{prob:5:a}
            Dar el valor numerico de la ley del oscurecimiento al borde $I(0,\mu)/I(0,1)$ de una atmosfera gris obtenida usando la aproximacion de las dos corrientes (representacion de $I(\tau,\mu)$ segun la parte b del ejercicio 1).
        \end{ppart}

        \begin{ppart}\label{prob:5:b}
            Dar el valor numerico de la ley del ordenamiento al borde de una atmosfera gris obtenido con la funcion de Hopf en la aproximacion de labs (practica 4)    
        \end{ppart}
        
        \begin{ppart}\label{prob:5:c}
            Compare a) y b) con la ley del ordenamiento al borde de una atmosfera gris obtenida con la expreción exácta de la funcion de Hopf:

           
        \end{ppart}
            \begin{table}[h!]
            \centering
                \begin{tabular}{c|c} 
                     
                     $\mu$ & $I(0,\mu)/I(0,1)$ \\ [0.5ex] 
                     \hline
                     1.0 & 1.000   \\ 
                     0.9 & 0.939   \\
                     0.8 & 0.878  \\
                     0.7 & 0.816  \\
                     0.6 & 0.755  \\
                     0.5 & 0.692  \\
                     0.4 & 0.629  \\
                     0.3 & 0.565  \\
                     0.2 & 0.499  \\
                     0.1 & 0.429 \\
                     0.0 & 0.344 \\ [1ex] 
                     
                \end{tabular}
            \end{table}
        
    \end{problem}

    \begin{problem}\label{prob:6}
        Obtener la expresión del flujo caliente de una copa gaseosa plano - forodilo finito de profundidad óptica total $\tau_{\nu}$ y $S_{\nu}(\tau_{\nu}) = cte$, cuejo fondo $(\tau_{\nu}=\tau_{\nu})$ esto iluminado por una radiación $I_{\nu}(\mu) = cte$. donde las expreciones corespondientes a los como limites:

        \begin{ppart}\label{prob:6:a}
            copa opticamente delgado $(\tau_{\nu}<< 1)$
        \end{ppart}

        \begin{ppart}\label{prob:6:b}
            copa opticamente espeso $(\tau_{\nu}>>1)$    
        \end{ppart}
        
        
    \end{problem}
    
    \begin{problem}\label{prob:7}
        \begin{ppart}\label{prob:7:a}
            Otener la función de Hopf para el cono $M=1$ y $M=2$

            \begin{equation*}
                \begin{array}{l c l c l}
                    M=1 & \mu_{1}=1/\sqrt{3}  & a_{1} = a_{-1} = 1\\
                      & \mu_{-1}= -1/\sqrt{3}  &  \\
                       &         &             &   \\
                    M=2 & \mu_{1}=0.33998  & a_{1} = a_{-1} =0.65215\\
                      & \mu_{-1}= -\mu_{1}  &  \\
                    &         &             &   \\
                      & \mu_{2}=0.86114  & a_{2} = a_{-2} =0.34785\\
                      & \mu_{-2}= -\mu_{2}  &  
                \end{array}
            \end{equation*}
        \end{ppart}
        

        \begin{ppart}\label{prob:7:b}
            ¿Cual es el valor de la funcion de Hopf $q(\tau)$ para cada uno de las aproximaciones en $ \tau = 0 $ y $\tau = \infty$?    
        \end{ppart}

        \begin{ppart}\label{prob:7:c}
            Calcular el valor de la función $q(\tau)$ obteniendo en la oproximacon $M=2$ y compara con un valor exacto:

                \begin{table}[h!]
                \centering
                    \begin{tabular}{c |c l c|c} 
                         
                         $\tau$ & $q(\tau)$ &  &$\tau$ & $q(\tau)$ \\ [0.5ex] 
                         \hline
                         0.00 & 0.5774 &  &0.8 & 0.6935 \\ 
                         0.05 & 0.6108 &  & 1.0 & 0.6985\\
                         0.10 & 0.6279 &  & 1.5 & 0.7051\\
                         0.20 & 0.6496 &  & 2.0 & 0.7079\\
                         0.30 & 0.6634 &  & 3.0 & 0.7078\\
                         0.40 & 0.6731 &  & 4.0 & 0.7103\\
                         0.50 & 0.6802 &  & 0.5 & 0.71040\\
                         0.60 & 0.6858 &  & $\infty$ & 0.71045\\ [1ex] 
                         
                    \end{tabular}
                \end{table}
        \end{ppart}
        
        
    \end{problem}
    

\end{document}

