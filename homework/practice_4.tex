%% Preámbulo
%% ----------------------------------------------------------------------------
\documentclass[12pt,a4paper]{practice}
\usepackage[spanish]{babel}
\usepackage[utf8]{inputenc}
\usepackage[T1]{fontenc}
\usepackage{enumitem}
\usepackage{hyperref}
\usepackage{luximono}
\usepackage{textcomp}
\usepackage{graphicx}
\usepackage{multirow}
\usepackage{tabularx}
\usepackage{ragged2e}
\usepackage{amssymb}
\usepackage{amstext}
\usepackage{caption}
\usepackage{charter}

%% Settings
%% ----------------------------------------------------------------------------
% hyperref
\hypersetup{
    pdftitle={Atm\'{o}sferas Estelares - Pr\'{a}ctica 1},
    pdfauthor={Mart\'{i}n Josemar\'{i}a Vuelta Rojas},
    pdfpagelayout=OneColumn,
    pdfnewwindow=true,
    pdfdisplaydoctitle=true,
    pdfstartview=XYZ,
    plainpages=false,
    unicode=true,
    bookmarksnumbered=true,
    bookmarksopen=true,
    bookmarksopenlevel=3,
    breaklinks=true,
    colorlinks=true,
    pdfborder={0 0 0}
}

% graphicx
\graphicspath{{resources/img/}}

% caption
\captionsetup{
    labelfont=bf,
    textfont=it,
    justification=centering,
    width=0.9\textwidth,
}

% tabularx rule width
\setlength{\arrayrulewidth}{1.5pt}

%% Definicion de comandos
%% ----------------------------------------------------------------------------
\makeatletter

%% Fuente de ancho fijo
\renewcommand{\ttdefault}{lmtt}
\renewcommand{\spanishtablename}{Tabla}

\makeatother

%% Documento
%% ----------------------------------------------------------------------------
\begin{document}
    %% Titulo, autor y resumen --------------------------------------------------
    \logo{unmsm.png}
    \university{
        Universidad Nacional Mayor de San Marcos\\
        {
            \scriptsize{
                \textit{
                    \textup{
                        Universidad del Perú, Decana de América}
                    }
            }
        }
    }
    \course{Atmósferas Estelares}
    \title{Práctica N\textdegree\ 4}
    \maketitle
    \begin{problem}
    Determinar la estructura de la atmósfera de una estrella cuyo parámetro fundamentales son los siguientes:

        \begin{table}[!h]
            \centering
            \begin{tabularx}{0.5\textwidth}{ *{3}{>{\Centering}X} }
                \hline
                $T_{eff}$ & $\log(g)$ & $\bar{\mu}$
                %!-> BEGIN monkeypatch to manage spacings in table header
                \rule{0pt}{2.6ex}\rule[-1.2ex]{0pt}{0pt}\\
                & & \\[-1.05em]\hline
                & & \\[-1.05em]
                %!-> END monkeypatch to manage spacings in table header
                18 000 & 4.0 &  0.666 \\
                20 000 & 4.0 &  0.658 \\
                22 500 & 4.0 &  0.655 \\
                25 000 & 4.0 &  0.653 \\
                \hline
            \end{tabularx}
        \end{table}

        Para el celaculo suponer:

        \begin{itemize}
            \item Que la distribución de la temperatura en función de la profundidad óptica de Rosseland está dada por la distribución de temperatura de una atmosfera gris.

            \item Que para la temperatura efectiva dado, se pueda utilizar un valor medio constante del peso molecular $\mu$
        \end{itemize}

        En este trabajo se debe tabular:

        \begin{itemize}
            \item $\tau$: Profundidad óptica de Rosseland.
            \item $T(\tau)$: Temperatura en $k$.
            \item $\bar{\varkappa}(\tau)$: Coeficiente de $\mu_{0}$ de Rosseland
            \item $P_{g}(\tau)$: Presión del gas en $dym/cm^2$
            \item $P_{r}(\tau)$: Presión de radiación en $dym/cm^2$
            \item $\rho(\tau)$: Densidad en $gr/cm^3$
            \item $h(\tau)$: Altura atmosférica en cm.
        \end{itemize}

        \begin{table}[!h]
            \centering
            \begin{tabularx}{0.5\textwidth}{ *{3}{>{\Centering}X} }
                \hline
                $log(g)$ & $a$ & $b$
                %!-> BEGIN monkeypatch to manage spacings in table header
                \rule{0pt}{2.6ex}\rule[-1.2ex]{0pt}{0pt}\\
                & & \\[-1.05em]\hline
                & & \\[-1.05em]
                %!-> END monkeypatch to manage spacings in table header
                3.90 & -0.3592 &  0.666 \\
                3.95 & -0.4019 &  0.658 \\
                4.00 & -0.4673 &  0.655 \\
                4.05 & -0.4948 &  0.653 \\
                \hline
            \end{tabularx}
        \end{table}
    \end{problem}
\end{document}
