%% Preámbulo
%% ----------------------------------------------------------------------------
\documentclass[12pt,a4paper]{practice}
\usepackage[spanish]{babel}
\usepackage[utf8]{inputenc}
\usepackage[T1]{fontenc}
\usepackage{enumitem}
\usepackage{hyperref}
\usepackage{luximono}
\usepackage{textcomp}
\usepackage{graphicx}
\usepackage{multirow}
\usepackage{tabularx}
\usepackage{ragged2e}
\usepackage{amssymb}
\usepackage{amstext}
\usepackage{caption}
\usepackage{charter}
\usepackage{mathabx}

%% Settings
%% ----------------------------------------------------------------------------
% hyperref
\hypersetup{
    pdftitle={Atm\'{o}sferas Estelares - Pr\'{a}ctica 1},
    pdfauthor={Mart\'{i}n Josemar\'{i}a Vuelta Rojas},
    pdfpagelayout=OneColumn,
    pdfnewwindow=true,
    pdfdisplaydoctitle=true,
    pdfstartview=XYZ,
    plainpages=false,
    unicode=true,
    bookmarksnumbered=true,
    bookmarksopen=true,
    bookmarksopenlevel=3,
    breaklinks=true,
    colorlinks=true,
    pdfborder={0 0 0}
}

% graphicx
\graphicspath{{resources/img/}}

% caption
\captionsetup{
    labelfont=bf,
    textfont=it,
    justification=centering,
    width=0.9\textwidth,
}

% tabularx rule width
\setlength{\arrayrulewidth}{1.5pt}

%% Definicion de comandos
%% ----------------------------------------------------------------------------
\makeatletter

%% Fuente de ancho fijo
\renewcommand{\ttdefault}{lmtt}
\renewcommand{\spanishtablename}{Tabla}

\makeatother

%% Documento
%% ----------------------------------------------------------------------------
\begin{document}
    %% Titulo, autor y resumen --------------------------------------------------
    \logo{unmsm.png}
    \university{
        Universidad Nacional Mayor de San Marcos\\
        {
            \scriptsize{
                \textit{
                    \textup{
                        Universidad del Perú, Decana de América}
                    }
            }
        }
    }
    \course{Atmósferas Estelares}
    \title{Práctica N\textdegree\ 2}
    \maketitle
     \begin{problem}\label{prob:1}

        \begin{ppart}\label{prob:1:a}
        Suponiendo $\mu = 2.5$ (peso molecular medio) y $T = 10 K$,
        calcular cual fue el radio probable de la nube prOtotoestelar que dio origen al sol. Suponen que ello tenía simetría esférico.
        \end{ppart}

        \begin{ppart}\label{prob:1:b}
        Teniendo en cuenta las mismas suposiciones y los mismos valores de $\mu$ y de $T$ de la parte $\ref{prob:1:a}$, Calcular los radios de las nubes protoestelares de las estrellas de la secuencia principal de tipos espectrales BO, AO y FO. Para este cálculo utilizar la relación masa-luminosidad

        $$
            L/L_{\odot} = 1.2(M/M_{\odot})^{4.0}
        $$

        y recordar que $L_{\odot} = 3.83 * 10^{33} \log /seg, M_{\odot} = 1.99*10^{33} g, M_{bol} = 4.77$. Usar las tablas de la práctica 1 donde se tiene $M_{v},CB=f(T.Sp.)$.
        \end{ppart}
    \end{problem}

    \begin{problem}\label{prob:2}
    Suponiendo que el campo de radiación en una nube protoestelar fuera isótropo, determinar cuál es la relación crítica entre la presión de radiación y la presión del gas cuyas masas totales son las del ejercicio $\ref{prob:1:a}$ y $\ref{prob:1:b}$. Suponer $\mu =2.5$. Tener en cuenta que $P_{r}$ (presión de radiación) $ = \frac{4\sigma}{C3}*T^{4}; \sigma$ :constante de Stefan Boltzmamm y C: la velocidad de la luz.
    \end{problem}

    \begin{problem}\label{prob:3}
    Mostrar que en una estrella esférica, las capas externas al radio r no influyen en el cálculo de la gravedad en r.
    \end{problem}

    \begin{problem}\label{prob:4}
        \begin{ppart}\label{prob:4:a}
        Obtener una expresión de la presión central $P_{c}$ de una estrella en función de la masa total M y de su radio R.
        \end{ppart}

        \begin{ppart}\label{prob:4:b}
        Obtener una expresión de $P_{c}$ en función de la densidad central $\rho_{\odot}$ de una estrella.
        \end{ppart}

        \begin{ppart}\label{prob:4:c}
        Obtener una expresión poco la temperatura media $\bar{T}$ de una estrella en función de la masa M y de R.
        \end{ppart}

        \begin{ppart}\label{prob:4:d}
        Obtener  una expresión poco la presión medio $\bar{P}$ de una estrella en función de M y de R.
        \end{ppart}

        \begin{ppart}\label{prob:4:e}
        Obtener $\bar{P}$  usando la luz de los gases perfectos.
        \end{ppart}

        \begin{ppart}\label{prob:4:f}
        Comparar $\bar{P}$ con $P_{c} = P_{c} (M,R)$
        \end{ppart}
    \end{problem}

    \begin{problem}\label{prob:5}
    Dos valores para $P_{c}(M,R)$, $\bar{P}(M,R)$, $\bar{\rho}(M,R)$ y $\bar{T}(M,R)$ para las estrellas de la tabla \ref{table:p5_table}. Usar $\mu = 0.6$
    \end{problem}

    \begin{problem}\label{prob:6}
    Comparar $\bar{P}$ y $\bar{P}$ del ejercicio \ref{table:p5_table} con la densidad medio $\bar{\rho}_{\Earth}$
    \end{problem}

    \begin{problem}\label{prob:7}
        \begin{ppart}\label{prob:7:a}
        Usando la ecuación del transporte radioactivo de energía en el interior de una estrella, tener una relación masa - radio - luminosidad.
        \end{ppart}

        \begin{ppart}\label{prob:7:b}
        Sabiendo que cada clase de luminosidad se caracteriza por una gravedad superficial aproximadamente constante, transformar la relación obteniendo en \ref{prob:7:a} en relación masa - luminosidad para cada clase de luminosidad.
        \end{ppart}
    \end{problem}

    \begin{problem}\label{prob:8}
    La luminosidad de una estrella puede representarse también por la relación siguiente:
            $$
            L = \frac{V.E}{t}
            $$

        donde:  L: luminosidad (erg/cm)
                V: volumen
                E: energía radioactiva por unidad de volumen
                T: tiempo que un fotón requiere poco recomen una distancia promedio equivalente al radio R de la estrella.
        $t= 3R^{2}\bar c$
        l: camino libre medio de un fotón
        l $\alpha T^{3.5}* \rho^{-2}$ si opacidad de Kramees
        l $\alpha \rho^{-1}$ si opacidad de electrones
        $E = a.T^{4}$   a: cte de radiación $= \frac{4 \rho}{c} $

        La ecuación de estado de las estrellas puede ser:
        P $\alpha \rho^{T}$ para masas pequeñas e intermedias.
        P $\alpha T^{4}$ para grandes masas.

        Sabiendo que
        P $\alpha M/R^{3}$ y P $\alpha M^{2}/R^{4}$

    Encontrar las cuatro expresiones posibles de L $\alpha L(M,R)$ combinando las diferentes masas de l y P. Compara las expresiones obtenidas con la relación masa - luminosidad empírica útil para estrellas de masas intermedios utilizado en el ejercicio \ref{prob:1:b}.  Decir que combinación de ecuaciones de estado y opacidades combinan para estas estrellas.
    \end{problem}

    \begin{problem}\label{prob:9}
    Obtener la distribución de la temperatura y de la presión en función de r de una envoltura estelar. Usar opacidades de Kramers y $\Phi = 10^{-6}$, donde:
    $$
        \Phi = \frac{3 \varkappa_o}{4ac(4\pi)^{3}}\cdot \left(\frac{k}{GH}\right)^{7.5}\cdot \frac{LR^{0.5}}{M^{5.5}}\cdot \frac{1}{\mu^{7.5}}
    $$

    $$
        \varkappa = \varkappa_o \cdot \rho\cdot T^{3.5}\ (cm^{2}/ gm)
    $$
    Calcular para $\chi$= r/R = 0.01,0.1,0.2,0.3,0.4 y 0.5.

    \end{problem}

        \begin{table}
            \centering
            \caption{
                Relación de $M/M_{\odot}$ y $R/R_{\odot}$ con el tipo espectral \\ (Problema \ref{prob:5})
            }\label{table:p5_table}
            \begin{tabularx}{\textwidth}{ *{3}{>{\Centering}X} }
                \hline
                T. Sp.    &  $M/M_{\odot}$    &  $R/R_{\odot}$
                %!-> BEGIN monkeypatch to manage spacings in table header
                \rule{0pt}{2.6ex}\rule[-1.2ex]{0pt}{0pt}\\
                & \\[-1.05em]\hline
                & \\[-1.05em]
                %!-> END monkeypatch to manage spacings in table header
                  O5 v  &  39.8  &     17.8  \\
                  B0 v  &  17.8  &      7.4   \\
                  B5 v  &   6.5  &      3.8   \\
                  A0 v  &   3.2  &      2.5   \\
                  A5 v  &   2.1  &     1.74   \\
                  F0 v  &   1.7  &     1.35   \\
                  F5 v  &   1.3  &     1.20   \\
                  G0 v  &   1.1  &     1.05   \\
                  G5 v  &  0.93  &     0.93   \\
                  K0 v  &  0.78  &     0.85   \\
                  K5 v  &  0.69  &     0.74   \\
                  M0 v  &  0.47  &     0.63   \\
                \hline
            \end{tabularx}
        \end{table}
\end{document}
